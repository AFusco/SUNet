\documentclass[12pt,a4paper]{report}
\usepackage[italian]{babel}
\usepackage[utf8]{inputenc}
\usepackage{newlfont}
\usepackage{csquotes}
\usepackage[
backend=biber,
style=alphabetic,
sorting=ynt
]{biblatex}

\addbibresource{mybibliography.bib}
\renewcommand{\baselinestretch}{1.5}

\textwidth=450pt
\oddsidemargin=0pt

\begin{document}

% INTESTAZIONE
\begin{titlepage}
\begin{center}
    {{\Large{\textsc{
        Alma Mater Studiorum $\cdot$ Università di Bologna
    }}}}
    \rule[0.1cm]{15.8cm}{0.1mm}
    \rule[0.5cm]{15.8cm}{0.6mm}
{\small{\bf 
    SCUOLA DI INGEGNERIA E ARCHITETTURA\\
    Corso di Laurea in Ingegneria Informatica
}}
\end{center}
\vspace{15mm}
\begin{center}
{\LARGE{\bf
    Determinazione della confidenza di mappe depth tramite Deep Learning
}}\\
\vspace{3mm}
\end{center}
\vspace{40mm}
\par
\noindent
\begin{minipage}[t]{0.47\textwidth}
{\large{\bf Relatore:\\
Chiar.mo Prof.\\
Mattoccia Stefano}}
\end{minipage}
\hfill
\begin{minipage}[t]{0.47\textwidth}\raggedleft
{\large{\bf Presentata da:\\
Fusco Alessandro}}
\end{minipage}
\vspace{20mm}
\begin{center}
{\large{\bf 
    Sessione di ottobre\\
    2016/2017
}}
\end{center}
\end{titlepage}


\chapter{Introduzione}
Nel vasto panorama del digitale, uno degli argomenti più caldi degli anni
recenti è l'intelligenza artificiale (AI), che sta ricevendo sempre più
attenzione sia dal mondo accademico che da quello industriale. 

In particolare il Machine Learning (apprendimento automatico), ovvero quella
branca dell'intelligenza artificiale che si occupa di fornire ai calcolatori
l'abilità di apprendere senza essere stati esplicitamente programmati, ha
subito recentemente un boom esponenziale.

Seppur le basi matematiche del Machine Learning siano state definite e studiate
lungo il corso del ventesimo secolo, solo di recente
\cite{krizhevsky2012imagenet} le si sono potute applicare efficientemente a
scopi pratici, grazie al notevole miglioramento delle tecnologie di calcolo
parallelo necessarie.

Tra le tante applicazioni di Machine Learning, emerge tra le altre quella
relativa alla computer vision.  Nel momento in cui si vede necessaria
un'interpretazione ad alto livello di quelli che sono i vari bit di
informazione di un'immagine o di un video, quando è necessario definire,
rilevare o ricostruire proprietà visuali d'interesse inesprimibili
proceduralmente, lì i mondi della visione artificiale e dell'apprendimento
automatico si vanno a fondere.

La visione stereo è quella branca della computer vision che si occupa della
ricostruzione di scene tridimensionali tramite l'acquisizione di immagini
bidimensionali, tentando di ricostruire l'informazione perduta nel processo di
acquisizione. 

L'esigenza di accuratezza nell'impiego di algoritmi stereo (detti di stereo
matching) ha portato necessità della creazione di misure di confidenza, ovvero
indici che riescano a valutare a monte la correttezza di un processo di
ricostruzione 3D arbitrario.

I metodi di misura di misura di confidenza attuali sono molteplici; alcune
strategie possono utilizzare nozioni geometriche del sistema stereoscopico;
metodi più complessi, quali alberi di classificazione, possono combinare le
varie strategie per valutare quale sia la più efficace in un determinato
contesto.  Tecniche più recenti basate su Deep Learning, anche se
computazionalmente più costose, hanno dimostrato però di portare a risultati di
qualità notevolmente superiore, in quanto riescono a riconoscere pattern ad
alto livello dei possibili problemi che possono indurre un algoritmo stereo a
riportare risultati errati, quale la presenza di superfici riflettenti o
occlusioni.

Questa tesi vuole essere uno studio su come applicare tecniche di deep learning
al fine della determinazione di una misura di confidenza "full-resolution",
ovvero calcolata analizzando l'intera immagine stereo.

\chapter{Sistemi di visione stereo}

\medskip
 
\printbibliography


\end{document}
